%
% File acl2018.tex
%
%% Based on the style files for ACL-2017, with some changes, which were, in turn,
%% Based on the style files for ACL-2015, with some improvements
%%  taken from the NAACL-2016 style
%% Based on the style files for ACL-2014, which were, in turn,
%% based on ACL-2013, ACL-2012, ACL-2011, ACL-2010, ACL-IJCNLP-2009,
%% EACL-2009, IJCNLP-2008...
%% Based on the style files for EACL 2006 by 
%%e.agirre@ehu.es or Sergi.Balari@uab.es
%% and that of ACL 08 by Joakim Nivre and Noah Smith

\documentclass[11pt,a4paper]{article}
\usepackage[hyperref]{acl2018}
\PassOptionsToPackage{breaklinks}{hyperref}
\usepackage{times}
\usepackage{graphicx}
\usepackage{latexsym}

\usepackage{todonotes}
\usepackage{booktabs}

\usepackage{algorithm}
\usepackage[noend]{algpseudocode}
\usepackage{amsmath}
\usepackage{amsfonts}
\usepackage{stmaryrd}
\usepackage{hyperref}
\usepackage[T1]{fontenc}
\usepackage{url}

\usepackage{xcolor}
\newcommand{\chris}[1]{{\textcolor{red}{\bf [{\sc chris:} #1]}}}
\newcommand{\henry}[1]{{\textcolor{blue}{\bf [{\sc henry:} #1]}}}

\aclfinalcopy % Uncomment this line for the final submission
% \def\aclpaperid{***} %  Enter the acl Paper ID here

%\setlength\titlebox{5cm}
% You can expand the titlebox if you need extra space
% to show all the authors. Please do not make the titlebox
% smaller than 5cm (the original size); we will check this
% in the camera-ready version and ask you to change it back.

\newcommand\BibTeX{B{\sc ib}\TeX}

\title{Generating High-Quality Surface Realizations Using Data Augmentation and Factored Sequence Models}

\author{wut wut}

\date{}

\begin{document}
\maketitle
\begin{abstract}
This work presents a new state of the art in reconstruction of surface realizations from obfuscated text. We identify the lack of sufficient training data as the major obstacle to training high-performing models, and solve this issue by generating large amounts of synthetic training data. We also propose preprocessing techniques which make the structure contained in the input features more accessible to sequence models. Our models were ranked first on all evaluation metrics in the English portion  of the 2018 Surface Realization shared task. 

% Using a pointer network, data augmentation and structuring the input data intelligently

\end{abstract}

\section{Introduction}

% New Paragraph: Background
% 		○ Broad and specific background
% 			§ What is this for our topic?
% 			§ Broad background is NLG
% 				□ NLG is
% 			§ Specific is surface realisation / abstract representation to text eg. E2E, WebNLG, AMR-to-text
% 				□ Surface realisation is

% New Paragraph: Unknown/Problem
% 		○ Problems of previous work and unknown factors
% 			§ This particular shared task of UD tree to text
% 				□ The best way to convert UD tree to text has not been determined
% 				□ The problem bears some similarities to AMR-to-text

% New paragraph: Question/Purpose of study
% 		○ Addition made by our research
% 			§ We assess the how best to model surface realisation from UD trees as tested using automated and human evaluation
% 			§ We examined the relative importance of additional data, ordering/dependency information, delemma dict

% New paragraph: Experimental approach
% 		○ State clearly the approach taken toward this addition
% 			§ We modelled the task as seq2seq problem using Neural networks with copy attention

% New paragraph: Results/Conclusion
% 			§ Highest results in the shared task


Contextualized Natural Language Generation (NLG) is a long-standing goal of Natural Language Processing (NLP) research. The task of generating text, conditioned on knowledge about the world, is applicable to almost any domain. However, despite recent advances in specific domains, NLG models still produce relatively low quality outputs in many settings. Representing the context in a consistent manner is still a challenge: how can we condition output on a stateful structure such as a graph or a tree?

% \citep{Colin2016TheData,novikova2017e2e} need citation for AMR shared task still
Several shared tasks have recently explored NLG from inputs with graph-like structures; RDF triples \citep{Colin2016TheData}, dialogue act-based meaning representations \citep{Novikova2017TheGeneration} and abstract meaning representations \citep{May2017}. In each of these challenges, the input has structure beyond simple linear sequences; however, to date, the top results in these tasks have consistently been achieved using relatively standard sequence-to-sequence models. 

The \textbf{surface realization} task is a conceptually simple challenge: given shuffled input, where tokens are represented by their lemmas, parts of speech, and dependency features, can we train a model to reconstruct the original text? A model that performs well at this task is likely to be a good starting point for solving more complex tasks, such as NLG from Resource Description Framework (RDF) graphs or Abstract Meaning Representation (AMR) structures. In addition, training data for the surface realization task can also be generated in a fully-automated manner.

% cite factored models in this paragraph
In this work, we show that training dataset size may be the major obstacle preventing current sequence-to-sequence models from doing well at NLG from structured inputs. Although inputting the structures themselves is theoretically appealing \cite{Tai2015ImprovedNetworks}, in many domains it may be enough to use sequential inputs by flattening structures, and providing structural information via input factors, as long as the training dataset is sufficiently large. By augmenting training data using a large corpus of unannotated data, we obtain a new state of the art in the surface realization task using off-the-shelf sequence to sequence models. 

In addition, we show that information about the output word order, implicitly available from parse features, provides essential information about the word order of correct output sequences, confirming that structural information cannot be discarded without a large drop in performance.

The main contributions of this work are:

\begin{enumerate}
  \item We show how training datasets can be augmented with synthetic data
  \item We apply preprocessing steps to simplify the universal dependency structures, making the structure more explicit
  \item We evaluate pointer models for the surface realization task 
\end{enumerate}


\begin{table*}[ht!]
\centering
\begin{tabular}{p{0.15\linewidth}p{0.45\linewidth}p{0.15\linewidth}p{0.15\linewidth}}
\toprule
 \textsc{Feature} & \textsc{Description} & \textsc{Vocabulary Size} & \textsc{Embedding Size}  \\ \midrule
 lemma & the lemma of the surface word & 30004 & 300 \\
 XPOS & the English part-of-speech label & 53 & 16 \\
 position & the position in the sequence & 103 & 25 \\
 UPOS & the universal part-of-speech label & 20 & 8 \\
 head position & the position of the head word according to the dependency parser & 100 & 25 \\
 deprel & the dependency relation label according to the dependency parser & 51 & 15 \\
\bottomrule
\end{tabular}
\caption{The features used in the factored models, along with the number of possible values the feature may take, and the respective embedding size.}
\label{tab:features}
\end{table*}
\section{The Surface Realization Shared Task}
% UD treebanks to text, effectively the opposite of the CoNLL 2017 and 2018 shared tasks
% Deep task removes "functional words (in particular, auxiliaries, functional prepositions and conjunctions) and surface-oriented morphological information" and aims to be comparable to the kind of output that might be generated by a data-to-text system

% here describe the SR shared task
In the \textbf{shallow} track of the 2018 surface realization (SR) shared task, inputs consist of tokens from a universal dependency (UD) tree provided in the form of lemmas. The original order of the sequence is obfuscated by random shuffling\footnote{The task organizers also introduced a \textbf{deep} task, but since ours was the only submission to the deep task, we save our discussion of this task for future work.}.

Models are evaluated on their ability to reconstruct the original, unshuffled input which generated the features. In order to do this, models must make use of structural information in order to reorder the tokens correctly as well as part-of-speech and/or dependency parse labels in order to restore the correct surface realization of lemmas. Note that we focus upon the English sub-task, where word order is critical because of the typologically analytic nature of English, however, for other languages, restoring word order may be less important, while deriving surface realizations from lemmas may be much more challenging. 


% possibly remove description of the deep task if we're not really going to talk about it much / at all in this paper
% And the deep task which removes additional tokens from the UD tree such as functional words and surface-oriented morphological information. The deep task replicates intermediary output from a data-to-text system. 

% add table with an example instance 

\section{Datasets}
% 	- Started out with the training data
% 	- Got wikitext 103
% 		○ Parsed it using udpipe
% 		○ Filtered for sentences with 95% vocabulary overlap
% 	- Combined the two datasets together. Tried use any upsampling, didn't see any improvement
% 		○ We probably try this extensively enough we only tried N and N/2, where N is num_filtered_wikitext_lines / num_srst_training_data_lines
% 	- Shuffled each conll dependency tree 
% 	- Sorted based a random depth first search
% 		○ (this could be improved by have a consistent approach to the random ordering as done in neural AMR paper)
% 	- Parsed the conll lines and appended the pos and dependency information to the lemma as features
% 	- The target sentences were tokenized using the NLTK port of the moses tokenizer with aggressive hyphen splitting


\subsection{Augmenting Training with Synthetic Datasets}

To augment the SR training data, we used sentences from the WikiText corpus \citep{Merity2016PointerModels}. Each of these sentences was parsed using UDPipe \cite{udpipe:2017} to obtain the same features provided by the SR organizers. We then filtered this data, keeping only sentences with at least 95\% vocabulary overlap with the in-domain SR training data. Note that the input vocabulary for this task is word lemmas, so at least 95\% of the tokens in each instance in our additional training data are lemmas which are also found in the in-domain data. The order of tokens in each instance of this additional dataset is then randomly shuffled to simulate the random input order in the SR data. 

% here put stats about the size of the additional dataset
We thus obtain 642,960 additional training instances, which are added to the 12,375 instances supplied by the SR shared task organizers.




\begin{table*}[ht!]
\centering
\begin{tabular}{p{0.1\linewidth}p{0.15\linewidth}p{0.15\linewidth}p{0.1\linewidth}p{0.2\linewidth}p{0.1\linewidth}p{0.15\linewidth}}
\toprule
 \textsc{position} & \textsc{lemma} & \textsc{XPOS} & \textsc{UPOS} & \textsc{head position} & \textsc{deprel} \\ \midrule
 1 & learn & VERB & VB & 2 & acl \\
 2 & lot & NOUN & NN & 4 & nsubj \\
 3 & there & PRON & EX & 4 & expl \\
 4 & be & VERB & VBZ & 0 & root \\
 5 & about & ADP & IN & 8 & case \\
 6 & a & DET & DT & 2 & det \\
 7 & . & PUNCT & . & 4 & punct \\
 8 & Chernobyl & PROPN & NNP & 1 & obl \\
 9 & to & PART & TO & 1 & mark \\
\bottomrule
\end{tabular}
\caption{An example from the training data, containing all features we use as input factors.}
\label{tab:input-data-example}
\end{table*}

\section{Features}

\subsection{Leveraging Structured Features}

% New paragraph: Sort lemmas use a depth first search through the parse tree  cite  Konstas2017NeuralGeneration
Because we have the dependency parse features for each input, some information about word order is implicitly available from the parse information, but discovering the structural relationship between the dependency parse features and the order of words in the output sequence is likely to be challenging for our sequence to sequence model. Therefore, we construct the original parse tree from the dependency features, and perform a depth-first search to sort and reorder the lemmas. This is similar to the linearization step performed by \citet{Konstas2017NeuralGeneration}, the main difference being we randomly choose between child nodes instead of using a predetermined order based on edge types.

% On the embedding sizes of the different features


%  Appended suggested forms to the sequence using a delemma dict 
% 	- In addition to the sequence of lemmas we added possible forms the lemmas could take
% 		○ Did this by constructing a dictionary using lemmas + xpos tag -> form
% 		○ Possible forms had the xpos tag and dependency id as features
% Question: is there a special token between the actual sequence and the mapped lemmas?
In order to further augment the available context, we experiment with adding potential delemmatized forms for each input lemma. The possible forms for each lemma were found by creating a map from $ (\mathbf{lemma}, \mathbf{xpos}) \rightarrow \mathbf{form}$, using the WikiText dataset. For each input lemma and xpos, we then check for the pair in the map -- if it exists, the corresponding form is appended to the sequence. This makes forms available to the pointer model for copying. 

For some lemma, xpos pairs there are multiple potential forms. When this occurs we add all potential forms to the input sequence. The mapping was found to cover 98.9\% of cases in the development set. 

\subsection{Factored Inputs}

Factored models were introduced by Alexandrescu et al. \shortcite{Alexandrescu:2006:NLM} as a means of including additional features beyond word tokens into neural language models. The key idea is to create a separate embedding representation for each feature type, and to concatenate the embeddings for each  input token to create its dense representation. Sennrich et al. \shortcite{SennrichH16:factors} showed that this technique is quite effective for neural machine translation, and some recent work, such as Hokamp \shortcite{Hokamp:2017} has successfully applied this technique to related sequence generation tasks. 

The embedding $ e_{j} $ for each input token $ x_{j} $ with factors $ F $ is created as in  Eq.~\ref{eq:factored_input}:

\begin{equation}
    e_{j} = \bigparallel_{k=1}^{|F|} \mathbf{E}_{k} x_{jk} 
    \label{eq:factored_input}
\end{equation}

\noindent where $ \bigparallel $ indicates vector concatenation, $ \mathbf{E}_{k} $ is the embedding matrix of factor $ k $, and $ x_{jk} $ is a one hot vector for the $k$-th input factor. Table \ref{tab:features} lists each of the factors used in our models, along with its corresponding embedding size. The embedding size of 300 for the lemma is set in configuration, while the embedding sizes of the other features are set heuristically by OpenNMT-py, using the heuristic $ |embedding_{k}| = |V_{k}|^{0.7} $, where $ |V_{k}| $ is the vocabulary size of feature $ k $. Table \ref{tab:input-data-example} gives an example from the training data with actual instantiations of each of the features. 
% "-feat_vec_exponent [0.7] If -feat_merge_size is not set, feature embedding sizes will be set to N^feat_vec_exponent where N is the number of values the feature takes."

\section{Model}

% look up who to cite for coverage
Models were trained using the OpenNMT-py toolkit \citep{Klein2017}. The model architecture is a 1 layer bidirectional recurrent neural network (RNN) with long short-term memory (LSTM) cells \citep{Hochreiter1997} and attention \citep{Luong2015EffectiveTranslation}. The model has 450 hidden units in the encoder and decoder layers, and 300 hidden units in the word embeddings which are learned jointly across the whole model. Dropout of 0.3 is applied between the LSTM stacks. We use a coverage attention layer \citep{Tu2016ModelingTranslation} with lambda value of 1.

% look up citation for sgd
The models are trained using stochastic gradient descent with learning rate 1. A learning rate decay of 0.5 is applied at each epoch once perplexity does not decrease on the validation set. Models were trained for 20 epochs. Output was decoded using beam search with beam size 5. Unknown tokens were replaced with the input token that had the highest attention value at that time step \citep{Vinyals2015}. Output from the epoch checkpoint which performed best on the development set was chosen for test set submission. 

The exploration and choice of hyperparameters was aided by the use of Bayesian hyperparameter optimization platform SigOpt\footnote{\url{https://sigopt.com/}}. 
% \citep{SigOpt}.


% 	- Model is pointer network with coverage and a whole bunch of other model options
% 		○ Final model limited vocab to most frequent 30k in combined 
% 		○ For pure SR shared task train we started setting 15k most frequent
% 		○ Presumably we ought to be able to list what size the actual vocabulary is / would have been

% model illustration here 

% New paragraph: We used OpenNMT + pointer network

% New paragraph: give specific options used and citations

% Use of opennmt pointer network


\section{Experiments}
% Details of model training options
% -share_embeddings -layers 1 -epochs 20 -copy_attn -word_vec_size 300 -rnn_size 450 -coverage_attn -copy_attn_force
%  LR of 1 and Decay of 0.5 after first epoch which doesn't improve on devset perplexity
%  Everything else defaults from http://opennmt.net/OpenNMT-py/options/train.html

% Decoder options 
%   - Decoder is beam size 5, and it replaces unknown tokens with the most probable attention token

We experiment with many different combinations of input features and training data, in order to understand which elements of the representation have the largest impact upon performance. 

% give details about how training configuration is different for the different model types, and approximately how long training takes. 
We limit vocabulary size during training to enable the pointer network to generalize to unknown tokens at test time. When using just the SR training data we train word embeddings for the 15,000 most frequent tokens from a possible 23,650 unique tokens. When using the combined SR training data and filtered WikiText dataset we use the 30,000 most frequent tokens from a possible 106,367 unique tokens.

We trained on a single Tesla K40 GPU. Training time was approximately 1 minute per epoch for the SR data and 1 hour per epoch for the combined SR data and filtered WikiText.


\section{Results}

% 
We report results using automated evaluation metric BLEU \citep{Papineni:2002:BMA:1073083.1073135}. On the test set we additionally report the NIST \citep{Przybocki2009} score and the normalized edit distance (DIST).

\begin{table}[!ht]
\centering
\begin{tabular}{p{0.6\linewidth}p{0.2\linewidth}}
\toprule
\textsc{System} & \textsc{BLEU} \\ \midrule
SR Baseline & 21.27 \\
SR + delemma suggestions & 23.75 \\
SR + delemma suggestions + linearization & 43.11 \\
SR + delemma suggestions + linearization + additional data & 68.86 \\
\bottomrule
\end{tabular}
\caption{Ablation study with BLEU scores for different configurations on the shallow task development set}
\label{tab:ablation-results}
\end{table}

Table \ref{tab:ablation-results} presents the results of the surface realization experiments. We observe three main components that drastically improve performance over the baseline model:

\begin{enumerate}
    \item augmenting the training set with more data
    \item reordering the input using the dependency parse features
    \item providing potential forms via the delemmatization map
\end{enumerate}

Table \ref{tab:srst-official-results} gives the official SR 2018 results from task organizers. Our system, which corresponds to the best configuration from Table \ref{tab:ablation-results} was ranked first across all metrics. 

% Here add results from task organizers
\begin{table}[!ht]
\centering
\begin{tabular}{lccc}
\toprule
\textsc{Team ID} & \textsc{BLEU} & \textsc{DIST} & \textsc{NIST} \\ \midrule
1 (Ours) & \textbf{69.14} & \textbf{80.42} & \textbf{12.02} \\
2 & 28.09 & 70.01 & 9.51 \\ 
3 & 8.04 & 47.63 &  7.71 \\
4 & 66.33 & 70.22 & \textbf{12.02} \\
5 & 50.74 & 77.56 & 10.62 \\
6 & 55.29 & 79.29 & 10.86 \\
7 & 23.2 & 51.87  & 8.86 \\
8 & 29.6 & 65.9  & 9.58  \\
\midrule
AVG & 41.3 & 67.86 & 10.15 \\
\bottomrule
\end{tabular}
\caption{Official results of the surface realization shared task using BLEU, DIST and NIST as evaluation metrics.}
\label{tab:srst-official-results}
\end{table}


\section{Related Work}
% background and related work

% NLG with graph inputs
% Synthetic data in NMT, distant supervision, semi-supervised learning
The surface realization task bears the closest resemblance to the SemEval 2017 shared task AMR-to-text \citep{May2017}. Our approach to data augmentation and preprocessing uses many insights from Neural AMR \citep{Konstas2017NeuralGeneration}. Traditional data-to-text systems use a rule based approach \citep{Reiter:2000:BNL:331955}.
% The Universal Dependency treebanks V2.0 contains 12,374 CoNLL-U, text pairs. 


% WebNLG has 18,102 (a larger 40k dataset was released after the task) and E2E has 42,061 data, text pairs training data, this shared task had only 12,374

% Neural amr paper cites Sennrich: "paired training procedure is largely inspired by Sennrich et al. (2016)." 

% We use the WikiText-103 dataset. It is parsed with UDPipe cite(udpipe:2017). It is filtered for sentences with a token overlap with the SR training data vocab of 95%. This gives us an additional 642,960 CoNNL-U, text pairs.

\section{Conclusion}

The main takeaway from this work is that data augmentation improves performance on the surface realization task. Although unsurprising, this result confirms that sufficient data is needed to achieve reasonable performance, and that flattened structural information such as dependency parse features is insufficient without additional preprocessing to reduce the complexity of the input. The surface realization task is ostensibly quite simple, thus it is surprising that baseline sequence to sequence models, which perform well in other tasks such as machine translation, cannot solve this task. We hypothesize that the lemmatization and shuffling of the input does not provide sufficient information to reconstruct the input. In sequences longer than a few words, there is likely to be significant ambiguity without additional structural information such as parse features. However, reconstructing the original sequence from unprocessed, flattened parse information alone is unrealistic using standard encoder-decoder models. 

In future work, we plan to explore more challenging variants of this task, while also experimenting with models that do not require feature-specific preprocessing to make use of rich structural information in the input. 



% \section*{Acknowledgments}

% The acknowledgments should go immediately before the references.  Do not number the acknowledgments section ({\em i.e.}, use \verb|\section*| instead of \verb|\section|). Do not include this section when submitting your paper for review.

% \citep{Kingma2014}

% include your own bib file like this:
%\bibliographystyle{acl}
%\bibliography{acl2018}
\bibliography{mendeley_v2,acl2018}
\bibliographystyle{acl_natbib}



\end{document}
